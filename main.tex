\documentclass[12pt,cmcyralt]{article}
\usepackage{cmap}     % поиск в PDF
\usepackage[T2A]{fontenc}   % кодировка
\usepackage[utf8]{inputenc}   % кодировка исходного текста
\usepackage[english,russian]{babel} % локализация и переносы
\usepackage{amsmath}                         %Подключает математический пакет
\usepackage{amsfonts}                          %Математические шрифты
 \usepackage{amsmath,amsthm,amssymb} 
\usepackage{color}                                      % Для цветного текста 
\usepackage[utf8]{inputenc}

\title{Лабораторная работа №1}
\author{Подготовлено Дарьей Вершицкой }
\date{Апрель 2022}

\begin{document}

\maketitle

\section{Условие}
Вычислить интеграл 
$$\oint_{C} {e^z \over {(z-1)^2(z+2)} } dz $$
в следующих заданиях контура:
 $1) \ |z-i| = 2;  $
 $2)\ |z+2-i| = 3$
 
 \section{Решение}
 Особые точки подынтегральной функции: i кратности 2, -2.
\\ 1). В указанную область попадает точка i
\\ 
$$\oint_{C} {e^z \over {(z-1)^2(z+2)} } dz = \oint_{|z-i| = 2} {{e^z\over {z+2}}\over(z-i)^2} dz = 2\pi i f \prime (i) = 2\pi i( \frac{e^z(z+2) - e^z}{(z+2)^2} )|_i = 2\pi i \frac{e^i(1+i)}{(i+2)^2} $$
\\ 2). В указанную область попадают точки i, -2
$$\oint_{C} {e^z \over {(z-1)^2(z+2)} } dz = \oint_{C_1} {e^z \over {(z-1)^2(z+2)} } dz + \oint_{C_2} {e^z \over {(z-1)^2(z+2)} } dz,$$
где $C_1 и C_2 $ 
- границы непересекающихся областей точек i и -2;

\begin{gathered}
\oint_{C} {e^z \over {(z-1)^2(z+2)} } dz = \oint_{C_1} {{e^z\over {z+2}}\over(z-i)^2} dz + \oint_{C_2} {{e^z\over {(z-i)^2}}\over z+2} dz = 2\pi i \frac{e^i(1+i)}{(i+2)^2} +2\pi i \frac{e^z} {(z-i)^2} |_{-2} = \\ = 2\pi i \frac{e^i(1+i)}{(i+2)^2} +  2\pi i  \frac{e^{-2}}{(i+2)^2} = \frac{2\pi i}{(i+2)^2}(e^i(1+i) + e^{-2})
\end{gathered}


\end{document}
